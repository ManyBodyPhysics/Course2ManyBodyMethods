\documentclass[prc,amsart,english]{revtex4}
\usepackage[T1]{fontenc}       % DC-fonts
\begin{document}
\title{Training in Advanced Low Energy Nuclear Theory: Report on Nuclear Talent Course on Many-body Methods for Nuclear Physics}
\maketitle
The present document aims at giving a summary of the 
Nuclear Talent course on Many-body methods held at the premises of GANIL, Caen, France, from July 6 to July 24 2015. 

\section{Report on Nuclear Talent Course on Many-body Methods for Nuclear Physics}

\subsection{Aims and Learning Outcomes}
The aim of the course is to acquire the fundamental concepts of the nuclear
many-body  physics as well as to learn solving complicated nuclear
many-body problems,  using advanced methods beyond mean field
approximations.  In this course the focus was on Breuckner
Hartree-Fock theory, Coupled Cluster theory, Green's function theory and
their applications. Hartree-Fock theory, full configuration interaction theory and
many-body perturbation theory were also discussed during the first
week.  The students implemented and derived these methods in their
development of a program applied to studies of infinite neutron matter
using a cartesian basis. Based on the results, the most likely outcome
of this course will be a scientific article where coupled cluster theory
at the level of doubles excitations is benchmarked against Green's
function up to third order algebraic diagrammatic construction [ADC(3)]. 
The course did also focus on how to write a scientific report via a final
assignment which will be graded. A special emphasis was put on how 
to develop a large numerical project and how to benchmark the results.

All teaching material was available before the course started at the github address \url{https://github.com/NuclearTalent/Course2ManyBodyMethods}. A simple update on a daily basis gave the students the latest versions of the lecture notes in several formats, PDF files, HTML files with interactive Python programs and iPython notebooks. 
The material can also be accessed at \url{http://nucleartalent.github.io/Course2ManyBodyMethods/doc/web/course.html}.
The material, with several source files,  is freely available and can be used as a resource for new Talent courses and/or for self studies. 
The students which were not familiar with second quantization and basic many-body topics were asked to use the introductory material which was made available well in advance. This material was used by the more advanced students as well. 

\subsection{Detailed Course Content}

The course contained material on both fundamental physical properties of many-body systems and the technicalities of calculations methods.
The detailed content was 
\begin{itemize}
\item Basic many-body physics, hamiltonians,
setup of Slater determinants and single-particle basis.
Rehearsal of basic many-body physics and second quantization
\item Hartree-Fock theory and many-body perturbation theory
\item Full configuration interaction theory
\item Bruekcner Hartree-Fock theory
\item Coupled cluster theory
\item Green's function theory
\item Concept of propagators and spectral distributions
\item Dispersive optical model and correlations and experiment
\item Properties of infinite matter with an emphasis on neutron star physics
\item Simple properties of nuclear forces
\item How to build a scientific project and write a good scientific report
\end{itemize}

The course ended with a final assignment. 
The students could select whether they wanted to apply Brueckner Hartree-Fock theory, Coupled Cluster theory or Green's function theory. The final projects
are listed at \url{http://nucleartalent.github.io/Course2ManyBodyMethods/doc/web/course.html}.
The final assignment  is graded
with marks A, B, C, D, E and failed for Master students and passed/not passed for PhD students.
Out of 25 students, seven students indicated that they would hand in the final report, with deadline September 30.  Five handed in their final reports for grading and eventual credit transfer. Of these three students reported on different Green's function projects (either RPA response or binding for nuclear matter) and two submitted a joint report on Coupled Cluster.

\subsection{Learning outcomes}  The main learning outcome within the time span of three weeks, was to have the students
understand the fundaments of nuclear many-body physics as well as to master the basics of methods like Hartree-Fock theory, Brueckner Hartree-Fock theory, Coupled Cluster theory and Green's function theory. The links between theory and experiment were emphasized, in particular the links between dispersive optical potential models, spectral functions and experiment. 
The students had to write a program which implemented the Hartree-Fock method in a cartesian basis for neutron matter. Based on this program, about half the students opted for developing a Coupled Cluster program with doubles excitation only whereas the remaining students
developed a Green's function codes for RPA and the ADC(3). All students were able to develop a Hartree-Fock program for infinite matter using a simpler model for the nuclear forces, the so-called Minnesota potential. And most students developed either a Coupled Cluster code or a Green's function code. 

  
\subsection{Teaching}
The course was taught as an intensive  course of duration of three weeks, with a
total time of 45 h of lectures, 45 h of exercises and  a final assignment of 2 weeks of work.
The total load is approximately 170 hours, corresponding to  {\bf 7 ECTS} in Europe.
The days were organized as follows: 9-12 lectures, time for exercises with assistance (including lunch)
till 18 (3 hours of allocated exercise sessions per day).
The course was held at the premises of GANIL, Caen, France, from
July 6 to July 24 in 2015.




\subsection{Teachers and expenses}
Carlo Barbieri (University of Surrey), Wim Dickhoff (Washington University, St Louis), Gaute Hagen (Oak Ridge National Laboratory), Morten Hjorth-Jensen (Michigan State University and University of Oslo), and Artur Polls (University of Barcelona). Marek Ploszajczak (Ganil) and Francesca Gulminelli (University of Caen) were the local organizers. 
Ganil covered all expenses except travel for all students (except two that were supported by FUSTIPEN) and two teachers (Barbieri and Polls).
Dickhoff, Hagen, and Hjorth-Jensen had their local expenses covered by FUSTIPEN. The students used the university dormitories. Lunches and dinners were provided by Ganil. 
All travels were covered by the home institutions. 


\subsection{Participants and their home institution and nationalities}

The target group for the Nuclear Talent courses is Master of Science students, PhD students and early post-doctoral fellows.
There were  experimentalists among the applicants. Out of 34 applicants, 26 students were selected. Priority was given
to Master of Science students and early PhD students. The applicants who were not admitted, were either post-doctoral 
fellows or PhD students on the verge of finishing their theses. One student had to cancel the participation two weeks before the course began.
In total 25 students attended the course for its full duration. Three of the accepted students were experimentalists. 

The students were expected to have operating programming skills in Fortran/C++/Python and knowledge of
quantum mechanics at an intermediate level, with basic knowledge of 
many-body physics. It was a rather heterogeneous group, with some students being rather knowledgeable in advanced quantum mechanics
and programming, while other students had a more limited background. Irrespective of this, all students were able to solve the pairing 
model using many-body perturbation theory and/or Coupled Cluster theory or Green's function theory. Most students were also able to 
perform Hartree-Fock calculations in a cartesian basis for neutron matter
knowledge.  Most students were able to implement coupled cluster theory at the level of doubles excitations or Green's function theory at the more complete ADC(3) approximation. 

Of the 25 students, five were Master of Science students and one was a post-doctoral fellow at Ganil. Of the PhD students, the majority
were in their first two years. Eleven nationalities (and three continents) and 16 different institutions/affiliations 
were represented.
The students and their respective institutions and nationalities  are listed in the table below. 
\begin{table}[hbtp]
  \begin{ruledtabular}
    \begin{tabular}{|c||l|l|l|}
     Name & Level &Institution &Nationality  \\\hline
David Arturo Amor Quiroz & PhD & Mexico City & Mexico  \\
Mackenzie Atkinson & PhD& Washington University & USA  \\
Pawel Baczyk & PhD& Warsaw University & Poland \\
Matthew Barton & PhD& Surrey & UK \\
Nathan Brady  & MSc& Univ. of Texas AM Commerce & USA \\
Bartholome Cauchois & PhD & GANIL & France \\
Sijie Dai & PhD& Beijing  University & China \\
Quentin Deshayes & PhD& LPC Caen & France \\
Dong Ding &PhD   & Washington University & China \\
Guoxiang Dong & PostDoc & GANIL & China \\
Alexandru Dumitrescu & PhD& NIPNE Bucharest & Romania \\
Quentin Fable & PhD& GANIL & France \\
Guan Jian Fu & PhD& Shanghai Jiao Tong University & China \\
Maciek Konieczka &MSc&  Warsaw University & Poland \\
Simon Lecluse & MSc & KU Leuven & Belgium \\
Justin Lietz & PhD& MSU & USA \\
Alexis Mercenne & MSc & GANIL & France \\
Samuel Novario & PhD& MSU & USA \\
Nathan Parzuchowski &PhD & MSU & USA \\
Selen Saatchi & PhD& Middle East Tech. Inst. Ankara & Turkey \\
Robin Smith & PhD& University of Birmingham & UK \\
Matthias Verlinde & MSc & University of Leuven & Belgium \\
Herlik Wibowo & PhD & Western Michigan University & Indonesia \\ 
Qiang Wu & PhD & State Key Lab., Beijing University & China \\
Latsamy Xayavong & PhD & CENBG Bordeaux & Laos \\ 
    \end{tabular}
  \end{ruledtabular}
\end{table} 

On average, the students performed very well. It was an excellent  and very active group. 
Five students handed in final reports for grading and credit transfer. 
\section{Summary and recommendations}

%Overall, this course was a very positive experience for both teachers
%and students.  The support from GANIL, and its year-long experiences
%with running Talent course, was central to the success of the
%course. Of uttermost importance was Marek  Ploszajczak's help with
%all administrative matters, from housing to essentially all  practicalities.
%Without his help and his  enthusiastic support, it is unlikely that this
%course could ever have been organized. Marek was also present at all lectures and organized a
%lovely excursion to scenic places in Normady as well as social dinners for students and teachers. 
%Marek's help was highly appreciated by all of us, students and teachers alike. 

Overall, this course was a very positive experience for both teachers and students. 
The support from GANIL, and its year-long experiences with running Talent courses,
was central to the success of the course. Of uttermost importance was help from
the local organization with all administrative matters, from housing to essentially all  practicalities.
Without his help and the enthusiastic support, it is unlikely that this
course could ever have been organized. Local coordinators were also present at all lectures to
provide support and arranged for a lovely excursion to scenic places in Normady as well as social
dinners for students and teachers.  These activities contributed to a great learning environment
and had important positive effects on the interactions among students and teachers.
Help from Marek Ploszajczak was highly appreciated by all of us, students and teachers alike. 


Even though the students were exposed to quite some tough formalism in a short time period, and they all came with a rather heterogeneous background, they all managed to write codes to study infinite matter with either Hartree-Fock theory, Coupled Cluster theory or Green's function theory. Having well-defined projects prior to the beginning of a course plays a central role in the final success of the course. The interactions between students and teachers in the afternoon sessions
were a central part of this success. 
The reports
from the students are included as a separate file. 

\end{document}




